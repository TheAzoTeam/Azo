Repositório para a matéria de Técnicas de Programação, Universidade de Brasília-\/\+F\+GA, para refatoração do projeto {\itshape \hyperlink{namespace_azo}{Azo}}, que foi desenvolvido na matéria de Introdução à Jogos Eletrônicos, Universidade de Brasília.

\subsubsection*{Envolvidos no Projeto}

\paragraph*{Refatoração}


\begin{DoxyItemize}
\item Miguel Alves (Desenvolvedor -\/ UnB F\+GA);
\item Kamila Costa (Desenvolvedor -\/ UnB F\+GA);
\item Igor Aragão (Desenvolvedor -\/ UnB F\+GA);
\item Lucas Vitor (Desenvolvedor -\/ UnB F\+GA);
\item Samuel Borges (Desenvolvedor -\/ UnB F\+GA);
\item Lude Ribeiro (Desenvolvedor -\/ UnB F\+GA);
\item Eduardo Lima (Desenvolvedor -\/ UnB F\+GA);
\item Nathalia Lorena (Desenvolvedor -\/ UnB F\+GA);
\end{DoxyItemize}

\paragraph*{Projeto Original}


\begin{DoxyItemize}
\item Allan Jefrey (Desenvolvedor -\/ UnB F\+GA);
\item Hugo Alves (Desenvolvedor -\/ UnB F\+GA);
\item Roger Lenke (Desenvolvedor -\/ UnB F\+GA);
\item Murilo Oliveira (Músico -\/ UnB Darcy Ribeiro);
\item Marina Rebello (Designer -\/ UnB Darcy Ribeiro);
\item Thainá Ferreira (Designer -\/ UnB Darcy Ribeiro).
\end{DoxyItemize}

\subsubsection*{Introdução e Objetivos}

Aplicar sobre o projeto {\itshape \hyperlink{namespace_azo}{Azo}} as Técnicas de Programação aprendidas no decorrer da matéria Técnicas de Programação. Evoluir o código e arquitetura visando melhorar a qualidade do código e aplicar os conhecimentos adiquiridos.

\char`\"{}\+Azo é um jogo onde o jogador deve controlar três diferentes personagens em diferentes épocas, para corrigir a distorção nas linhas temporais!\char`\"{}

\subsubsection*{História}

No ano de 2097, uma filósofa humana e um cientista de uma raça alienígena discutem para saber quem veio primeiro\+: o ovo, ou a galinha. Sem chegar num consenso, ambos acordam em projetar uma máquina do tempo para que possam verificar por si mesmos. Infelizmente, os cálculos para a concepção da máquina estavam errados, o que causou uma explosão na máquina do tempo e distorceu as linhas temporais! Agora, três diferentes heróis serão escolhidos para coletar os fragmentos da máquina e corrigir o tempo, ou o universo estará condenado para sempre.

\subsubsection*{Características}


\begin{DoxyItemize}
\item Gênero\+: Runner/\+Plataforma.
\item Quantidade de jogadores\+: Single-\/player.
\item Quantidade de níveis\+: 1.
\item Personagens\+: 1.
\end{DoxyItemize}

\subsubsection*{Objetos}

Os objetos presentes no jogo são\+:
\begin{DoxyItemize}
\item Obstáculos de pulo.
\item Obstáculos para abaixar/deslizar.
\item Plataformas comuns.
\item Espinhos.
\item Fragmentos da máquina do tempo, que são coletáveis.
\end{DoxyItemize}

\subsubsection*{Controles}

O jogador pode\+:
\begin{DoxyItemize}
\item Saltar obstáculos (tecla \textquotesingle{}w\textquotesingle{}).
\item Deslizar por baixo de obstáculos (tecla \textquotesingle{}s\textquotesingle{}).
\item Selecionar opções nos menus (tecla \textquotesingle{}seta para direita\textquotesingle{} e \textquotesingle{}seta para esquerda\textquotesingle{}).
\item Ativar as opções selecionadas (tecla enter).
\end{DoxyItemize}

\subsubsection*{Dependências}

Para executar o jogo com sucesso é necessário possuir instaladas tais dependências\+:
\begin{DoxyItemize}
\item C\+Make 3.\+5.\+1
\item S\+DL 2
\item S\+D\+L\+\_\+image 2
\item S\+D\+L\+\_\+ttf 2
\item S\+D\+L\+\_\+mixer 2
\end{DoxyItemize}

\subsubsection*{Como executar}

No terminal do sistema operacional (Linux), utilize os comandos na pasta do clone do projeto\+: 
\begin{DoxyCode}
1 $ mkdir build
\end{DoxyCode}
 
\begin{DoxyCode}
1 $ cd build
\end{DoxyCode}
 
\begin{DoxyCode}
1 $ cmake ..
\end{DoxyCode}
 
\begin{DoxyCode}
1 $ make
\end{DoxyCode}
 
\begin{DoxyCode}
1 $ ./Azo
\end{DoxyCode}


Também é possível criar um instalador .deb para o projeto com os seguintes comandos\+: 
\begin{DoxyCode}
1 $ mkdir build
\end{DoxyCode}
 
\begin{DoxyCode}
1 $ cd build
\end{DoxyCode}
 
\begin{DoxyCode}
1 $ cmake ..
\end{DoxyCode}
 
\begin{DoxyCode}
1 $ make package
\end{DoxyCode}


O instalador estará localizado na pasta build. 